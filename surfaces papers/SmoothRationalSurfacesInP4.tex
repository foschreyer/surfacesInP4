\documentclass[twoside,12pt, leqno]{amsart}

\usepackage{amsmath,amscd,amsthm,amssymb,amsxtra,latexsym,epsfig,epic,graphics}
\usepackage[matrix,arrow,curve]{xy}
\usepackage{graphicx}
\usepackage{diagrams}
\usepackage{MnSymbol}
%\usepackage{pgfplots}
\usepackage{tikz}  %TikZ
\usepackage{color} 
\usetikzlibrary{arrows,calc} 
\usetikzlibrary{decorations.pathmorphing}
\usetikzlibrary{decorations.markings} 
\usetikzlibrary{decorations.pathreplacing} 
\usetikzlibrary{plothandlers}

%\usepackage{amsrefs}
%%%%%%%%%%%%%%%%%%%%%%%%%%%%%%%%%%%%%%%%%
%\textwidth16cm
%\textheig\codim20cm
%\topmargin-2cm
\oddsidemargin.8cm
\evensidemargin1cm


%%%%%Definitions
%\input {preamble}
\def\e{{\epsilon}}
\def\TU{{\bf U}}
\def\AA{{\mathbb A}}
\def\BB{{\mathbb B}}
\def\bB{{\mathbb B}}
\def\PP{{\mathbb P}}
\def\RR{{\mathbb R}}
\def\CC{{\mathbb C}}
\def\QQ{{\mathbb Q}}
\def\FF{{\mathbb F}}
\def\TT{{\mathbb T}}
\def\facet{{\bf facet}}
\def\image{{\rm image}}
\def\name{{\rm name}}
\def\cE{{\cal E}}
\def\cF{{\cal F}}
\def\cG{{\cal G}}
\def\cH{{\cal H}}

\def\sEnd{{\mathcal End}}

\def\cHom{{{\cal H}om}}
\def\fix#1{{\bf ***Fix:} #1 {\bf ***}}

\DeclareMathOperator{\rH}{{\rm H}}
%\DeclareMathOperator{\Ext}{{\rm Ext}}
\DeclareMathOperator{\Cliff}{{\rm Cliff}}
\DeclareMathOperator{\chara}{{char}}
\DeclareMathOperator{\hess}{{hess}}
\DeclareMathOperator{\rank}{{rank}}
\def\fC{{\mathfrak C}}
\def\Tr{{\rm Tr}}
\def\bC{{\mathbb C}}
\def\Gr{{\rm Gr}}
\def\CI{{\mathcal I}}
\def\CH{{\mathcal H}}
%\def\CCH{{\mathcal {CNT}}}
\def\CCH{{\mathcal {HC}}}
\def\rH{{\rm H}}

\def\all{{\{1,\ldots,2g+2\}}}
\def\soc{{\rm soc\,}}
\def\jacobian{{\rm Jac}}
\def\Rbar{{\overline R}}
\def\Ibar{{\overline I}}
\def\mm{{\frak m}}
\def\Trace{{\rm Tr}}

\def\CO{{\mathcal O}}
\def\CT{{\mathcal T}}
\def\CHom{{\mathcal Hom}}
\def\Spec{{{\rm Spec}\,}}
\def\cone{{{\rm cone}\,}}

\def\tR{{\widetilde R}}
\def\tI{{\widetilde I}}
\def\tJ{{\widetilde J}}
\def\tK{{\widetilde K}}
\def\tH{{\widetilde H}}
\def\tF{{\widetilde F}}



\def\Abar{{\overline A}}
\def\Rbar{{\overline R}}
\def\Ibar{{\overline I}}
\def\Jbar{{\overline J}}
\def\Kbar{{\overline K}}
\def\abar{{\overline \alpha}}
\def\bbar{{\overline \beta}}
\def\m{{\frak m}}
\def\Rbar{{\overline R}}

\def\gr{{\rm gr}}
\def\init{{\rm in}}

\def\frank#1{{\bf *** Frank:} #1 {\bf ***}}
\def\lbracket{{[\kern-1.5pt[}}
\def\rbracket{{]\kern-1.5pt]}}

\def\seq#1#2{{#1_{1},\dots,#1_{#2}}}
\def\ff#1{{f_{1},\dots, f_{#1}}}

\makeatletter
\def\Ddots{\mathinner{\mkern1mu\raise\p@
\vbox{\kern7\p@\hbox{.}}\mkern2mu
\raise4\p@\hbox{.}\mkern2mu\raise7\p@\hbox{.}\mkern1mu}}
\makeatother


%%%%%%%%%%%%%%%%%%Silvio's macros for the diagrams
\usepackage{times}
\newdimen\x \x=12pt

%\usepackage{mat\codimime}
\usepackage{color}

%\usepackage{color}
%\usepackage[usenames,dvipsnames,svgnames,table]{xcolor}

\usepackage[breaklinks,bookmarksopen,bookmarksnumbered,urlcolor=blue]{hyperref}
\hypersetup{colorlinks=true,backref=true,citecolor=blue}

%\pagestyle{MYheadings}
%\date{April 2013-December 2015}


\author[Frank-Olaf Schreyer]{Frank-Olaf Schreyer}
\address{Fachbereich Mathematik, Universit\"at des Saarlandes, Campus E2 4, D-66123 Saar\-br\"ucken, Germany}
\email{schreyer@math.uni-sb.de}

\title{Smooth rational surfaces in $\PP^{4}$}

\keywords{}
\begin{document}

\maketitle

\section{Introduction}

\section{Known Examples}

\subsection{Degenerate surfaces}
The plane $\PP^{2}$, $\PP^{1}\times \PP^{1}$ and the cubic surfaces $\PP^{2}(3;1^{6})$  are the smooth surfaces which do not span $\PP^{4}$. They have degree $1$, $2$ and $3$
respectively.

\subsection{Non-degenerate surfaces of degree $\le 7$.}

\begin{enumerate}
\item The cubic scroll $X\cong \PP^{2}(2;1) =\{\rank \begin{pmatrix} x_{0}& x_{1}&x_{3} \cr x_{1}& x_{2}& x_{4}\end{pmatrix} <2\} $

\item The del Pezzo surface of degree $4$, i.e. $\PP^{2}(3;1^{5)} = Q_{1}\cap Q_{2}$, a complete intersection of two quadrics.

\item The projected Veronese surface of degree $4$

\item The Castelnuovo surface $X \to \PP^{1}$ of degree $5$, a conic bundle over $\PP^{1}$

\item The Bordiga surface $\PP^{2}(4;1^{10})$ of degree $6$

\item The Ionescu-Okonek surface of $\PP^{2}(6;2^{6},5^{1})$ degree $7$ with with Betti table 
$$\begin{matrix}1 &- & -&-& -\cr 
                          - & - & - &- &-\cr 
                         - & 1& - &- &-\cr
                          - & 6 & 10& 5& 1 \cr \end{matrix}$$

\item The non-special Alexander surface $\PP^{2}(7;2^{10},1)$ of degree $8$ and sectional genus $\pi=5$ with Betti table
$$\begin{matrix}
      & 0 & 1 & 2 & 3 & 4\\
%     \text{total:} & 1 & 9 & 14 & 8 & 2\\
     0: & 1 & . & . & . & .\\
     1: & . & . & . & . & .\\
     2: & . & . & . & . & .\\
     3: & . & 5 & 4 & . & .\\
     4: & . & 4 & 10 & 8 & 2
     \end{matrix}$$

\item The Okonek surface $\PP^{2}(6;2^{4},1^{12})$ of degree $8$ and  sectional genus $\pi=7$ with Betti table
$$\begin{matrix}
 %      & 0 & 1 & 2 & 3\\
%      \text{total:} & 1 & 5 & 5 & 1\\
         1 & . & . & .\\
        . & . & . & .\\
        . & 1 & . & .\\
        . & 4 & 5 & 1
      \end{matrix}$$



\item The non-special Alexander surface $\PP^{2}(13;4^{10})$ of degree $9$ and sectional genus $\pi=6$ with Betti table
$$\begin{matrix}
%       & 0 & 1 & 2 & 3 & 4\\
       %     \text{total:} & 1 & 16 & 29 & 18 & 4\\
      1 & . & . & . & .\\
       . & . & . & . & .\\
       . & . & . & . & .\\
       . & . & . & . & .\\
       . & 15 & 26 & 15 & 3\\
       . & 1 & 3 & 3 & 1
      \end{matrix}$$

\item The speciality one Alexander surface $\PP^{2}(9;3^{6},2^{3},1^{6})$ of degree $9$ and sectional genus $\pi=7$ and Betti table
$$\begin{matrix}
%       & 0 & 1 & 2 & 3 & 4\\
%     \text{total:} & 1 & 9 & 15 & 9 & 2\\
       1 & . & . & . & .\\
       . & . & . & . & .\\
       . & . & . & . & .\\
       . & 3 & 1 & . & .\\
       . & 6 & 14 & 9 & 2
      \end{matrix}$$



\item The Ranestad surface $\PP^{2}(14;6,4^{9},2,1^{2})$ of degree $10$ and sectional genus $\pi=8$ and Betti table
$$\begin{matrix}
%       & 0 & 1 & 2 & 3 & 4\\
 %     \text{total:} & 1 & 14 & 24 & 14 & 3\\
       1 & . & . & . & .\\
        . & . & . & . & .\\
        . & . & . & . & .\\
        . & . & . & . & .\\
        . & 10 & 13 & 4 & .\\
        . & 4 & 11 & 10 & 3
      \end{matrix}$$
      
\item The Decker-Ein-Schreyer surface $\PP^{2}(9;3^{4},2^{7},1^{7})$ of degree $10$ and sectional genus $\pi=9$ with Betti table
 $$\begin{matrix}
%       & 0 & 1 & 2 & 3 & 4\\
 %     \text{total:} & 1 & 11 & 18 & 10 & 2\\
 1 & . & . & . & .\\
 . & . & . & . & .\\
 . & . & . & . & .\\
 . & 1 & . & . & .\\
 . & 10 & 18 & 10 & 2
      \end{matrix}$$
      
      
 \item The Ranestad surface $\PP^{2}(8;2^{12},1^{6})$ of degree $10$ and sectional genus $\pi=9$ and Betti table     
$$\begin{matrix}
        & 0 & 1 & 2 & 3 & 4\\
%       \text{total:} & 1 & 8 & 12 & 6 & 1\\
       0: & 1 & . & . & . & .\\
       1: & . & . & . & . & .\\
       2: & . & . & . & . & .\\
       3: & . & 2 & . & . & .\\
       4: & . & 5 & 9 & 3 & .\\
       5: & . & 1 & 3 & 3 & 1
       \end{matrix}$$


\item\label{schFamWiths=2} $\ge 3$ families of Schreyer surfaces of degree $11$ and sectional genus $\pi=10$ with Betti table
$$\begin{matrix}
       & 0 & 1 & 2 & 3 & 4\\
     % \text{total:} & 1 & 12 & 26 & 20 & 5\\
      0: & 1 & . & . & . & .\\
      1: & . & . & . & . & .\\
      2: & . & . & . & . & .\\
      3: & . & . & . & . & .\\
      4: & . & 5 & . & . & .\\
      5: & . & 7 & 26 & 20 & 5
      \end{matrix}$$
of type $\PP^{2}(18;6^{5},5^{5},2,1^{4})$, $\PP^{2}(15;5^{5},4^{4},3^{2},2,1^{3})$. The third family has as last adjoint surface the intersection of
$Y=(\PP^{1}\times \PP^{2})\cap Q \subset \PP^{5}$  with a quadric $Q$ and $X=Y(H+4R;4^{3},2^{2},1^{3})$
where $R$ denotes the ruling of $\PP^{1}\times \PP^{2}$. The conic fibration $Y \to \PP^{1}$ has 6 singular fibers.


\item\label{schFamWiths=3} $\ge 3$ families of Schreyer surfaces of  degree $11$ and sectional genus $\pi=10$ with Betti table
$$\begin{matrix}
       & 0 & 1 & 2 & 3 & 4\\
       %\text{total:} & 1 & 12 & 26 & 20 & 5\\
      0: & 1 & . & . & . & .\\
      1: & . & . & . & . & .\\
      2: & . & . & . & . & .\\
      3: & . & . & . & . & .\\
      4: & . & 5 & 1 & . & .\\
      5: & . & 8 & 26 & 20 & 5
      \end{matrix}$$
of type $\PP^{2}(13;4^{7},3^{4},2^{2},1^{2})$, $\PP^{1}\times \PP^{1}((9,9);4^{8},3,2^{3},1^{2})$ and $\PP^{2}(15;5^{6},4^{2},3^{2},2^{3},1^{2})$.
 
 \item $\ge 1$ families of Schreyer surfaces of  degree $11$ and sectional genus $\pi=10$ with Betti table
$$\begin{matrix}
       & 0 & 1 & 2 & 3 & 4\\
      \text{total:} & 1 & 14 & 28 & 20 & 5\\
      0: & 1 & . & . & . & .\\
      1: & . & . & . & . & .\\
      2: & . & . & . & . & .\\
      3: & . & . & . & . & .\\
      4: & . & 5 & 2 & . & .\\
      5: & . & 9 & 26 & 20 & 5
      \end{matrix}.$$
The adjunction leads to a quadric bundle of class $Y=2H-R$ in a scroll of type $S(2,1,1)$. The surface $X$ is the blow-up of $Y$ in $3^{4},2^{4},1$ points.
$X=Y(2H+2R;3^{4},2^{4},1)$.

 
  
  \item One family of Schreyer surfaces $\PP^{2}(12;4^{4},3^{5},2^{6})$ of  degree $11$ and sectional genus $\pi=10$ with Betti table
$$\begin{matrix}
        & 0 & 1 & 2 & 3 & 4\\
       \text{total:} & 1 & 15 & 29 & 20 & 5\\
       0: & 1 & . & . & . & .\\
       1: & . & . & . & . & .\\
       2: & . & . & . & . & .\\
       3: & . & . & . & . & .\\
       4: & . & 5 & 3 & . & .\\
       5: & . & 10 & 26 & 20 & 5
       \end{matrix}$$
%Could be in the boundary of the corresponding family in \ref{schFamWiths=3}, but is not.
      
\item   The Popsecu surface $\PP^{2}(9;3^{4},2^{7},1^{7})$ of degree $11$ and sectional genus $\pi=11$ with Betti table
$$\begin{matrix}
%       & 0 & 1 & 2 & 3 & 4\\
 %     \text{total:} & 1 & 10 & 14 & 6 & 1\\
 1 & . & . & . & .\\
 . & . & . & . & .\\
 . & . & . & . & .\\
 . & . & . & . & .\\
 . & 10 & 12 & 3 & .\\
 . & . & 2 & 3 & 1
      \end{matrix}$$

\item   The Popsecu surface of degree $11$ and sectional genus $\pi=11$ with Betti table
$$\begin{matrix}
       & 0 & 1 & 2 & 3 & 4\\
%      \text{total:} & 1 & 11 & 16 & 7 & 1\\
      0: & 1 & . & . & . & .\\
      1: & . & . & . & . & .\\
      2: & . & . & . & . & .\\
      3: & . & . & . & . & .\\
      4: & . & 10 & 13 & 4 & .\\
      5: & . & 1 & 3 & 3 & 1
      \end{matrix}$$

\item   The Popsecu surface of degree $11$ and sectional genus $\pi=11$ with Betti table and Tate resolution
$$\begin{matrix}
       & 0 & 1 & 2 & 3 & 4\\
%      \text{total:} & 1 & 12 & 19 & 10 & 2\\
      0: & 1 & . & . & . & .\\
      1: & . & . & . & . & .\\
      2: & . & . & . & . & .\\
      3: & . & . & . & . & .\\
      4: & . & 10 & 14 & 6 & 1\\
      5: & . & 2 & 5 & 4 & 1
      \end{matrix} \hbox{ and }
      \begin{matrix}
        & -1 & 0 & 1 & 2 & 3 & 4 & 5 & 6 & 7\\
  %     \text{total:} & 108 & 64 & 32 & 11 & 3 & 3 & 11 & 38 & 91\\
       -4: & 1 & . & . & . & . & . & . & . & .\\
       -3: & 107 & 64 & 32 & 11 & . & . & . & . & .\\
       -2: & . & . & . & . & 3 & 1 & . & . & .\\
       -1: & . & . & . & . & . & 2 & 1 & . & .\\
       0: & . & . & . & . & . & . & 10 & 38 & 91
       \end{matrix}
     $$
     respectively.

\item The vBothmer-Erdenberger-Ludwig surface $\PP^{2}(9;3,2^{14},1^{5})$ of degree $11$ and sectional genus $\pi=11$ with Betti table and Tate resolution
$$\begin{matrix}
       & 0 & 1 & 2 & 3 & 4\\
%      \text{total:} & 1 & 8 & 13 & 8 & 2\\
      0: & 1 & . & . & . & .\\
      1: & . & . & . & . & .\\
      2: & . & . & . & . & .\\
      3: & . & 1 & . & . & .\\
      4: & . & 5 & 4 & . & .\\
      5: & . & 2 & 9 & 8 & 2
      \end{matrix}
\hbox{ and  } 
\begin{matrix}
        & -1 & 0 & 1 & 2 & 3 & 4 & 5 & 6 & 7\\
%       \text{total:} & 108 & 64 & 32 & 11 & 3 & 4 & 12 & 38 & 91\\
       -4: & 1 & . & . & . & . & . & . & . & .\\
       -3: & 107 & 64 & 32 & 11 & . & . & . & . & .\\
       -2: & . & . & . & . & 3 & 1 & . & . & .\\
       -1: & . & . & . & . & . & 2 & 2 & . & .\\
       0: & . & . & . & . & . & 1 & 10 & 38 & 91
       \end{matrix}
$$
respectively.

\item\label{ARsurf} Six families of Abo-Ranestad surfaces of degree $12$ and sectional genus $\pi=13$ with Betti table
$$\begin{matrix}
       & 0 & 1 & 2 & 3 & 4\\
%%      \text{total:} & 1 & 9 & 18 & 13 & 3\\
      0: & 1 & . & . & . & .\\
      1: & . & . & . & . & .\\
      2: & . & . & . & . & .\\
      3: & . & . & . & . & .\\
      4: & . & 5 & . & . & .\\
      5: & . & 4 & 18 & 13 & 3
      \end{matrix}$$
      %({(4, 12, 13), 7, (12, 24, 13), 3, (12, 19, 8), 9, (7, 7, 1)} already there
of type $\PP^{2}(12;4^{5},2^{12},1^{4})$, $\PP^{2}(12;4^{4},3^{3},2^{9},1^{5})$, $\PP^{2}(12,4^{3},3^{6},2^{6},1^{6})$ and
$\PP^{2}(12;4^{2},3^{9},2^{3},1^{7})$. Further families:
%{(4, 12, 13), 7, (12, 24, 13), 3, (12, 19, 8), 9, (7, 7, 1)}
The fifth case has as last adjoint surface a quadric bundle of degree $8$ which is a divisor $Y$ of class $2H-2R$ on a scroll of degree $5$ (and type $S(2,2,1)$?) and
$X=Y(H+3R;3^{5},2^{4},7^{1})$. Finally, the
sixth case has as last adjoint surface a quadric bundle of degree $9$ which is a is a divisor $Y$ of class $2H-3R$ on a scroll of degree $6$ (perhaps of type $S(2,2,2)$) and
$X=Y(H+3R;3^{8},2,8^{1})$. 

\item\label{ARsurf10} One family of Abo-Ranestad surfaces $\PP^{2}(12;4^{5},2^{12},1^{4})$ of degree $12$ and sectional genus $\pi=13$ with Betti table
%{(4, 12, 13), 4, (12, 24, 13), 12, (12, 16, 5), 0, (4, 4, 1)}
$$
\begin{matrix}
       & 0 & 1 & 2 & 3 & 4\\
      \text{total:} & 1 & 10 & 19 & 13 & 3\\
      0: & 1 & . & . & . & .\\
      1: & . & . & . & . & .\\
      2: & . & . & . & . & .\\
      3: & . & . & . & . & .\\
      4: & . & 5 & 1 & . & .\\
      5: & . & 5 & 18 & 13 & 3
      \end{matrix}.$$
This family could be in the closure of the first family of \ref{ARsurf}.
\end{enumerate}

So at the moment there are at least $22+2+2+5=31$ components of the Hilbert scheme of surfaces in $\PP^{4}$ whose general elements correspond to non-degenerate  smooth rational surfaces.

{\bf Caveat}: Check that all are extendible to characteristic 0 and check whether family \ref{ARsurf10} lies in the closure of the first family of \ref{ARsurf}.


\end{document}

